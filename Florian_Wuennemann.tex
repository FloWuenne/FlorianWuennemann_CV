%!TEX TS-program = xelatex
%!TEX encoding = UTF-8 Unicode
% Awesome CV LaTeX Template for CV/Resume
%
% This template has been downloaded from:
% https://github.com/posquit0/Awesome-CV
%
% Author:
% Claud D. Park <posquit0.bj@gmail.com>
% http://www.posquit0.com
%
%
% Adapted to be an Rmarkdown template by Mitchell O'Hara-Wild
% 23 November 2018
%
% Template license:
% CC BY-SA 4.0 (https://creativecommons.org/licenses/by-sa/4.0/)
%
%-------------------------------------------------------------------------------
% CONFIGURATIONS
%-------------------------------------------------------------------------------
% A4 paper size by default, use 'letterpaper' for US letter
\documentclass[11pt,a4paper,]{awesome-cv}

% Configure page margins with geometry
\usepackage{geometry}
\geometry{left=1.4cm, top=.8cm, right=1.4cm, bottom=1.8cm, footskip=.5cm}


% Specify the location of the included fonts
\fontdir[fonts/]

% Color for highlights
% Awesome Colors: awesome-emerald, awesome-skyblue, awesome-red, awesome-pink, awesome-orange
%                 awesome-nephritis, awesome-concrete, awesome-darknight

\definecolor{awesome}{HTML}{00785c}

% Colors for text
% Uncomment if you would like to specify your own color
% \definecolor{darktext}{HTML}{414141}
% \definecolor{text}{HTML}{333333}
% \definecolor{graytext}{HTML}{5D5D5D}
% \definecolor{lighttext}{HTML}{999999}

% Set false if you don't want to highlight section with awesome color
\setbool{acvSectionColorHighlight}{true}

% If you would like to change the social information separator from a pipe (|) to something else
\renewcommand{\acvHeaderSocialSep}{\quad\textbar\quad}

\def\endfirstpage{\newpage}

%-------------------------------------------------------------------------------
%	PERSONAL INFORMATION
%	Comment any of the lines below if they are not required
%-------------------------------------------------------------------------------
% Available options: circle|rectangle,edge/noedge,left/right

\name{Florian}{Wünnemann}

\position{Postdoc in the group of Denis Schapiro}
\address{Im Neuenheimer Feld 130, 69120 Heidelberg, Germany}

\email{\href{mailto:flowuenne@gmail.com}{\nolinkurl{flowuenne@gmail.com}}}
\homepage{florianwuennemann.com}
\github{flowuenne}
\twitter{flowuenne}

% \gitlab{gitlab-id}
% \stackoverflow{SO-id}{SO-name}
% \skype{skype-id}
% \reddit{reddit-id}

\quote{\textbf{Research and scientific interests}: Spatial OMICS,
Cardiovascular disease, Human genetics, Bioinformatics, Machine
Learning, Computer vision}

\usepackage{booktabs}

\providecommand{\tightlist}{%
	\setlength{\itemsep}{0pt}\setlength{\parskip}{0pt}}

%------------------------------------------------------------------------------


\usepackage{float} \usepackage{multicol} \usepackage{colortbl} ¶
\arrayrulecolor{white} ¶ \usepackage{hhline} ¶
\definecolor{light-gray}{gray}{0.95}

% Pandoc CSL macros
\newlength{\cslhangindent}
\setlength{\cslhangindent}{1.5em}
\newlength{\csllabelwidth}
\setlength{\csllabelwidth}{2em}
\newenvironment{CSLReferences}[3] % #1 hanging-ident, #2 entry spacing
 {% don't indent paragraphs
  \setlength{\parindent}{0pt}
  % turn on hanging indent if param 1 is 1
  \ifodd #1 \everypar{\setlength{\hangindent}{\cslhangindent}}\ignorespaces\fi
  % set entry spacing
  \ifnum #2 > 0
  \setlength{\parskip}{#2\baselineskip}
  \fi
 }%
 {}
\usepackage{calc}
\newcommand{\CSLBlock}[1]{#1\hfill\break}
\newcommand{\CSLLeftMargin}[1]{\parbox[t]{\csllabelwidth}{\honortitlestyle{#1}}}
\newcommand{\CSLRightInline}[1]{\parbox[t]{\linewidth - \csllabelwidth}{\honordatestyle{#1}}}
\newcommand{\CSLIndent}[1]{\hspace{\cslhangindent}#1}

\begin{document}

% Print the header with above personal informations
% Give optional argument to change alignment(C: center, L: left, R: right)
\makecvheader

% Print the footer with 3 arguments(<left>, <center>, <right>)
% Leave any of these blank if they are not needed
% 2019-02-14 Chris Umphlett - add flexibility to the document name in footer, rather than have it be static Curriculum Vitae


%-------------------------------------------------------------------------------
%	CV/RESUME CONTENT
%	Each section is imported separately, open each file in turn to modify content
%------------------------------------------------------------------------------



\hypertarget{research-experience}{%
\section{\texorpdfstring{\faIcon{flask} Research
experience}{ Research experience}}\label{research-experience}}

\begin{cventries}
    \cventry{Postdoc in the Schapiro Lab}{University Hospital Heidelberg}{Heidelberg, Germany}{22-Jan-22 - ongoing}{\begin{cvitems}
\item Investigation of cellular neighbourhoods and tissue architecture in myocardial infarction models using spatial OMICS technologies
\end{cvitems}}
    \cventry{Postdoc in the Lettre lab}{Montreal Heart Institute}{Montreal, Canada}{18-Aug-22 - 21-Aug-22}{\begin{cvitems}
\item Projects focused on high-throughput CRISPR screens, polygenic risk scores, genetics of heart valve disease and development of single-cell screens to investigate human cellular traits.
\end{cvitems}}
    \cventry{Professionnel recherche niv. II}{Universite de Sherbrooke}{Remote work}{19-Jan-22 - 16-Jan-22}{\begin{cvitems}
\item Part of the GenAP initiative (www.genap.ca) as single-cell expert, to include single-cell tools into the GenAP2 platform. Development of Galaxy tools, Rshiny applications and docker containers for single-cell analysis.
\end{cvitems}}
    \cventry{Student Research assistant}{IEB, University of Münster}{Münster, Germany}{11-Sep-22 - 12-Jul-22}{\begin{cvitems}
\item Acquisti group: Analysis of genomes and metagenomes in the context of nutrient limitation and fertilization.
\end{cvitems}}
    \cventry{Student assistant}{IEB, University of Münster}{Münster, Germany}{11-Mar-22 - 11-May-22}{\begin{cvitems}
\item Bornberg-Bauer group
\end{cvitems}}
\end{cventries}

\hypertarget{education}{%
\section{\texorpdfstring{\faIcon{university}
Education}{ Education}}\label{education}}

\begin{cventries}
    \cventry{Ph.D. (Dr.rer.nat) - Life sciences}{University of Münster / CHU Sainte Justine Research Center}{Münster, Germany / Montreal, Canada}{Apr.2014 - Apr.2018}{\begin{cvitems}
\item Thesis title: The role of genetic factors in pathogenesis and progression of cardiac malformations
\end{cvitems}}
    \cventry{MSc in Life sciences}{University of Münster}{Münster, Germany}{Oct.2011 - Feb.2014}{\begin{cvitems}
\item Thesis title: Functional and genetic characterization of a novel arrhythmic syndrome
\end{cvitems}}
    \cventry{BSc in Life sciences}{University of Münster}{Münster, Germany}{Oct.2008 - Sep.2011}{\begin{cvitems}
\item Thesis title: Impact of nutrient limitation in insects: Comparative genomics of the pea aphid and the human body louse
\end{cvitems}}
\end{cventries}

\hypertarget{funding-history}{%
\section{\texorpdfstring{\faIcon{dollar-sign} Funding
History}{ Funding History}}\label{funding-history}}

\begin{cventries}
    \cventry{Postdoctoral Training (Canadian citizens and permanent residents) scholarship, Fonds de recherche Québec santé (FRQS)}{}{}{2019 - 2021}{}\vspace{-4.0mm}
\end{cventries}

\hypertarget{achievements-and-awards}{%
\section{\texorpdfstring{\faIcon{award} Achievements and
Awards}{ Achievements and Awards}}\label{achievements-and-awards}}

\begin{cvhonors}
    \cvhonor{}{Poster prize: Prix Fonds de recherche du Québec (FRQS) (Recherche fondamentale doctorale / Postdoc)}{Montreal,Canada}{2021/06}
    \cvhonor{}{Best oral presentation award, 22nd Montreal Heart Institute research day}{Montreal,Canada}{2019/06}
    \cvhonor{}{Best oral presentation award, 32nd student congress at the CHU Sainte-Justine}{Montreal, Canada}{2017/05}
    \cvhonor{}{Markwald award for best oral presentation, Weinstein Cardiovascular Development and Regeneration Conference 2016}{Durham, USA}{2016/05}
\end{cvhonors}

\hypertarget{invited-presentations}{%
\section{\texorpdfstring{\faIcon{file-powerpoint} Invited
Presentations}{ Invited Presentations}}\label{invited-presentations}}

\begin{cventries}
    \cventry{Molecular cartography helps reveal immune cell infiltration routes and their microenvironments during acute myocardial infarction.}{Resolve Biosciences - Immersive spatial experience 2022}{Heidelberg, Germany}{26.09 - 26.09}{}\vspace{-4.0mm}
\end{cventries}

\hypertarget{presentations}{%
\section{\texorpdfstring{\faIcon{file-powerpoint}
Presentations}{ Presentations}}\label{presentations}}

\begin{cventries}
    \cventry{Prioritization of genomic loci for coronary artery disease using targeted CRISPR screens for endothelial dysfunction}{American Society of Human Genetics (ASHG) Meeting 2019}{Houston, Texas, USA}{15.10.2019 - 19.10.2019}{}\vspace{-4.0mm}
    \cventry{Validation of genome-wide polygenic risk scores for coronary artery disease in French Canadians}{XXIIe Journée de la recherche ICM}{Montreal, Canada}{06.06.2019 - 06.06.2019}{}\vspace{-4.0mm}
    \cventry{Single cell landscape of mammalian heart maturation}{7th annual MGSE Symposium}{Münster, Germany}{21.03.2018 - 22.03.2018}{}\vspace{-4.0mm}
    \cventry{Identification of a novel marker for valve maturation: Loss of ADAMTS19 function causes progressive valve disease in mice and men}{American Society of Human Genetics (ASHG) Meeting 2017}{Orlando, Florida, USA}{18.10.2017 - 18.10.2017}{}\vspace{-4.0mm}
    \cventry{Heart valve dysfunction in men and mice is caused by loss of function mutations in Adamts19, a novel marker for valvular interstitial cells}{Congrès de la recherche des étudiantes des cycles supérieurs et des post-doctorants en recherche au CHU Sainte-Justine}{Montreal, Canada}{26.05.2017 - 26.05.2017}{}\vspace{-4.0mm}
    \cventry{Loss of Adamts19, a novel marker for valvular interstitial cell populations during valve maturation, causes aortic valve dysfunction}{Weinstein Cardiovascular Development and Regeneration Conference 2016}{Durham, North Carolina, USA}{18.05.2016 - 21.05.2016}{}\vspace{-4.0mm}
    \cventry{ACTransDB: An online database for Acanthamoeba castellani transcripts,}{Evolgen, collaborative meeting on genome evolution}{Ciążeń, Poland}{27.06.2012 - 28.06.2012}{}\vspace{-4.0mm}
    \cventry{Biogeochemistry meets molecular evolution via metagenomics: tracing nitrogen fluxes from ecosystems to genomes in microbial communities}{2nd Muenster graduate school evolution symposium}{Münster, Germany}{18.06.2012 - 19.06.2012}{}\vspace{-4.0mm}
\end{cventries}

\hypertarget{poster-presentations}{%
\section{\texorpdfstring{\faIcon{chalkboard-teacher} Poster
presentations}{ Poster presentations}}\label{poster-presentations}}

\begin{cventries}
    \cventry{CRISPR perturbations at many coronary artery disease loci impair vascular endothelial cell functions}{XXIIIe Journée de la recherche ICM}{Montreal, Canada}{17.06.2021 - 17.06.2021}{}\vspace{-4.0mm}
    \cventry{CRISPR perturbations at many coronary artery disease loci impair vascular endothelial cell functions}{Cold Spring Harbor Laboratories: The Biology of Genomes}{Virtual}{11.05.2021 - 14.05.2021}{}\vspace{-4.0mm}
    \cventry{A single-cell perspective on growth and maturation pathways in the mouse heart.}{Weinstein Cardiovascular Development and Regeneration Conference}{Nara, Japan}{16.05.2018 - 18.05.2018}{}\vspace{-4.0mm}
    \cventry{De novo mutation in SOX18 causes a novel form of Hypotrichosis-Lymphedema-Telangiectasia with severe vascular defects}{American Society of Human Genetics (ASHG) 2014}{San Diego, California, USA}{18.10.2014 - 22.10.2014}{}\vspace{-4.0mm}
    \cventry{Soil metagenomics to unravel the signature of fertilizers on the molecular composition of the bacterial ribosome}{42nd Annual Meeting of the Ecological Society of Germany, Austria and Switzerland 2012}{Lueneburg, Germany}{10.09.2012 - 14.09.2012}{}\vspace{-4.0mm}
\end{cventries}

\hypertarget{preprints}{%
\section{\texorpdfstring{\faIcon{scroll}
Preprints}{ Preprints}}\label{preprints}}

\hypertarget{bibliography}{}
\leavevmode\vadjust pre{\hypertarget{ref-wunnemann2021crispr}{}}%
\CSLLeftMargin{1. }%
\CSLRightInline{Wünnemann, F., Tadjo, T. F., Beaudoin, M., Lalonde, S.,
Lo, K. S., \& Lettre, G. (2021). CRISPR perturbations at many coronary
artery disease loci impair vascular endothelial cell functions.
\emph{bioRxiv}.}

\leavevmode\vadjust pre{\hypertarget{ref-churakov20204}{}}%
\CSLLeftMargin{2. }%
\CSLRightInline{Churakov, G., Kuritzin, A., Chukharev, K., Zhang, F.,
Wünnemann, F., Ulyantsev, V., \& Schmitz, J. (2020). A 4-lineage
statistical suite to evaluate the support of large-scale retrotransposon
insertion data to reconstruct evolutionary trees. \emph{bioRxiv}.}

\hypertarget{publications}{%
\section{\texorpdfstring{\faIcon{scroll}
Publications}{ Publications}}\label{publications}}

\hypertarget{bibliography}{}
\leavevmode\vadjust pre{\hypertarget{ref-Biermann2022-nm}{}}%
\CSLLeftMargin{1. }%
\CSLRightInline{Biermann, J., Melms, J. C., Amin, A. D., Wang, Y.,
Caprio, L. A., Karz, A., Tagore, S., Barrera, I., Ibarra-Arellano, M.
A., Andreatta, M., Fullerton, B. T., Gretarsson, K. H., Sahu, V.,
Mangipudy, V. S., Nguyen, T. T. T., Nair, A., Rogava, M., Ho, P., Koch,
P. D., \ldots{} Izar, B. (2022). Dissecting the treatment-naive
ecosystem of human melanoma brain metastasis. \emph{Cell},
\emph{185}(14), 2591--2608.e30.}

\leavevmode\vadjust pre{\hypertarget{ref-heckel2022triglyceride}{}}%
\CSLLeftMargin{2. }%
\CSLRightInline{Heckel, E., Cagnone, G., Agnihotri, T., Cakir, B., Das,
A., Kim, J. S., Kim, N., Lavoie, G., Situ, A., Pundir, S., et al.
(2022). Triglyceride-derived fatty acids reduce autophagy in a model of
retinal angiomatous proliferation. \emph{JCI Insight}, \emph{7}(6).}

\leavevmode\vadjust pre{\hypertarget{ref-audain2021integrative}{}}%
\CSLLeftMargin{3. }%
\CSLRightInline{Audain, E., Wilsdon, A., Breckpot, J., Izarzugaza, J.
M., Fitzgerald, T. W., Kahlert, A.-K., Sifrim, A., Wünnemann, F.,
Perez-Riverol, Y., Abdul-Khaliq, H., et al. (2021). Integrative analysis
of genomic variants reveals new associations of candidate
haploinsufficient genes with congenital heart disease. \emph{PLoS
Genetics}, \emph{17}(7), e1009679.}

\leavevmode\vadjust pre{\hypertarget{ref-wunnemann2020loss}{}}%
\CSLLeftMargin{4. }%
\CSLRightInline{Wünnemann, F., Ta-Shma, A., Preuss, C., Leclerc, S.,
Vliet, P. P. van, Oneglia, A., Thibeault, M., Nordquist, E., Lincoln,
J., Scharfenberg, F., et al. (2020). Loss of ADAMTS19 causes progressive
non-syndromic heart valve disease. \emph{Nature Genetics}, \emph{52}(1),
40--47.}

\leavevmode\vadjust pre{\hypertarget{ref-gould2019robo4}{}}%
\CSLLeftMargin{5. }%
\CSLRightInline{Gould, R. A., Aziz, H., Woods, C. E., Seman-Senderos, M.
A., Sparks, E., Preuss, C., Wünnemann, F., Bedja, D., Moats, C. R.,
McClymont, S. A., et al. (2019). ROBO4 variants predispose individuals
to bicuspid aortic valve and thoracic aortic aneurysm. \emph{Nature
Genetics}, \emph{51}(1), 42--50.}

\leavevmode\vadjust pre{\hypertarget{ref-luyckx2019copy}{}}%
\CSLLeftMargin{6. }%
\CSLRightInline{Luyckx, I., Kumar, A. A., Reyniers, E., Dekeyser, E.,
Vanderstraeten, K., Vandeweyer, G., Wünnemann, F., Preuss, C., Mazzella,
J.-M., Goudot, G., et al. (2019). Copy number variation analysis in
bicuspid aortic valve-related aortopathy identifies TBX20 as a
contributing gene. \emph{European Journal of Human Genetics},
\emph{27}(7), 1033--1043.}

\leavevmode\vadjust pre{\hypertarget{ref-wunnemann2019validation}{}}%
\CSLLeftMargin{7. }%
\CSLRightInline{Wünnemann, F., Sin Lo, K., Langford-Avelar, A.,
Busseuil, D., Dubé, M.-P., Tardif, J.-C., \& Lettre, G. (2019).
Validation of genome-wide polygenic risk scores for coronary artery
disease in french canadians. \emph{Circulation: Genomic and Precision
Medicine}, \emph{12}(6), e002481.}

\leavevmode\vadjust pre{\hypertarget{ref-preuss2017heart}{}}%
\CSLLeftMargin{8. }%
\CSLRightInline{Preuss, C., Wünnemann, F., \& Andelfinger, G. (2017). At
the heart of a complex disease {``molecular genetics of congenital heart
disease.''} \emph{eLS}, 1--9.}

\leavevmode\vadjust pre{\hypertarget{ref-gillis2017candidate}{}}%
\CSLLeftMargin{9. }%
\CSLRightInline{Gillis, E., Kumar, A. A., Luyckx, I., Preuss, C.,
Cannaerts, E., Van De Beek, G., Wieschendorf, B., Alaerts, M., Bolar,
N., Vandeweyer, G., et al. (2017). Candidate gene resequencing in a
large bicuspid aortic valve-associated thoracic aortic aneurysm cohort:
SMAD6 as an important contributor. \emph{Frontiers in Physiology},
\emph{8}, 400.}

\leavevmode\vadjust pre{\hypertarget{ref-wunnemann2016aortic}{}}%
\CSLLeftMargin{10. }%
\CSLRightInline{Wünnemann, F., Kokta, V., Leclerc, S., Thibeault, M.,
McCuaig, C., Hatami, A., Stheneur, C., Grenier, J.-C., Awadalla, P.,
Mitchell, G. A., et al. (2016). Aortic dilatation associated with a de
novo mutation in the SOX18 gene: Expanding the clinical spectrum of
hypotrichosis-lymphedema-telangiectasia syndrome. \emph{Canadian Journal
of Cardiology}, \emph{32}(1), 135--e1.}

\leavevmode\vadjust pre{\hypertarget{ref-preuss2016family}{}}%
\CSLLeftMargin{11. }%
\CSLRightInline{Preuss, C., Capredon, M., Wünnemann, F., Chetaille, P.,
Prince, A., Godard, B., Leclerc, S., Sobreira, N., Ling, H., Awadalla,
P., et al. (2016). Family based whole exome sequencing reveals the
multifaceted role of notch signaling in congenital heart disease.
\emph{PLoS Genetics}, \emph{12}(10), e1006335.}

\leavevmode\vadjust pre{\hypertarget{ref-wunnemann2016molecular}{}}%
\CSLLeftMargin{12. }%
\CSLRightInline{Wünnemann, F., \& Andelfinger, G. U. (2016). Molecular
pathways and animal models of hypoplastic left heart syndrome. In
\emph{Congenital heart diseases: The broken heart} (pp. 649--664).
Springer, Vienna.}

\leavevmode\vadjust pre{\hypertarget{ref-chetaille2014mutations}{}}%
\CSLLeftMargin{13. }%
\CSLRightInline{Chetaille, P., Preuss, C., Burkhard, S., Côté, J.-M.,
Houde, C., Castilloux, J., Piché, J., Gosset, N., Leclerc, S.,
Wünnemann, F., et al. (2014). Mutations in SGOL1 cause a novel
cohesinopathy affecting heart and gut rhythm. \emph{Nature Genetics},
\emph{46}(11), 1245.}

\hypertarget{computational-skills}{%
\section{\texorpdfstring{\faIcon{laptop-code} Computational
skills}{ Computational skills}}\label{computational-skills}}

\begin{itemize}
\tightlist
\item
  \textbf{General}: GWAS analysis, Exome/Genome variant calling, Plink,
  bedtools, Image analysis (Fiji, Napari, QuPath)
\item
  \textbf{R}: Rshiny application development, Rmarkdown, Package
  development, OMICS data analysis (RNA-seq, single-cell OMICS),
  reticulate
\item
  \textbf{Python}: Jupyter notebooks, basic computer vision
  applications, single-cell OMICS analysis
\item
  \textbf{Containers}: Docker container creation, Singularity usage,
  Nextflow workflow creation and execution
\item
  \textbf{Galaxy project}: Creation of galaxy tools and wrappers
\end{itemize}

\hypertarget{languages}{%
\section{\texorpdfstring{\faIcon{language}
Languages}{ Languages}}\label{languages}}

\begin{itemize}
\tightlist
\item
  German (mother-language)
\item
  English (fluent)
\item
  French (fluent)
\end{itemize}



\end{document}
